\section{Inledning}

\subsection{Parter}
ENVISIoN är en produkt som beställts av beställaren Rickard Armiento. Produkten har skapats av projektgruppen, redovisade under projektidentiteten, under handledning av handledare Johan Jönsson. Se projektidentitet för mer genomgående information om beställare och handledare.    

%Beställare: \LIPSkund \\
%Leverantör: Grupp 1.

\subsection{Projektets bakgrund} 
%Visualisering och simulering av beräkningsresultat är mycket väsentligt för förståelsen hos olika sorterts analyser. Syftet med ENVISIoN är att visualisera kvantmekaniska beräkningar för att underlätta analyseringen av resultaten. När mjukvaran är klar är syftet att den används i forskningssyften. 
Projektet skapades i samband med kandidatarbetet, kursen TFYA75. Som ett väsentligt examinerade moment ska projektet genomförande återspegla alla förutsättningar, krav och ansvarstaganden som råder under en formell anställning. Utveckling av ENVISIoN är en av flera projekt som kan väljas inom kursen. Produktens slutgiltiga syfte är att användas som ett forskningsverktyg for visualisering av kvantmekaniska beräkningar.      

\subsection{Syfte och mål}
Projektet syftar till att utveckla kreativiteten samt att ge färdigheter i fysikalisk tänkande och analys av teoretiska resultat. Projektet bedrivs realistiskt som en träning inför det kommande yrkeslivet. Resultatet av projektarbetet ska hålla hög vetenskaplig och teknisk kvalité och baseras på moderna kunskaper, dokumenteras i form av projekt-och tidsplan, krav-och designspecification samt i en teknisk/vetenskaplig rapport, presenteras muntligt, demonstreras och följas upp i en efterstudie. Målet är att i visualiseringsverktyget Inviwo utveckla ett system för visualisering av resultatet av elektronstrukturberäkningar. Att demonstrera systemetsfunktionalitet genom att använda det till att illustrera några befintliga beräkningsresultat.

\subsection{Användning}
Denna produkt kommer huvudsakligen användas vid Linköpings universitet för att analysera data från elektronstruktursberäkningar.

\subsection{Begränsningar}
I projektet kommer visualiseringsverktyget Inviwo och programmeringspråken Python och C++ användas. Det kommer inte utredas om det är bättre att använda andra verktyg.

