\section{Parsersystemet} 
Parserystemets uppgift är att omvandla information från VASP-filer till data i HDF5-format, som visualiseringssystemet kan använda. Parsersystemet är det delsystem i ENVISIoN som ser till att avläsa korrekt data från VASP-filer och spara denna data i en lämplig HDF5-filstruktur. Följande kapitel beskriver hur parsersystmet har implementerats, samt redogör bakgrundskunskaper om HDF5 och VASP.  

\subsection{Bakgrundskunskap}
För förståelse över hur parsersystemet implementerats krävs det lite bakgrundskunskaper om hur HDF5 är uppbyggt och vad VASP är.   
\subsubsection{VASP}
VASP är ett beräkningsprogam som använder sig av Hartree-Fock metoden eller täthetsfunktionalteori (DFT) för att approximera en lösning för Schrödingerekvationen för mångpartikelfallet \cite{Quick Start Guide}. VASP-filer kan delas upp i indatafiler och utdatafiler. I indatafiler anges information som användaren kan manipulera, dessa indatafiler styr hur beräkningarna ska utföras. Efter beräkningar genereras sedan ett antal utdatafiler som innehåller kalkylresultaterna. Varje datafil korresponderar till specifik information om systemet. Nedan återfinns några viktiga VASP-filer.   

\textbf{Utdatafiler:} 
\begin{itemize}
    \setlength\itemsep{0em}
    \item CHG innehåller data om laddningstäthet.
    \item DOSCAR innehåller data om tillståndstäthet.
    \item EIGENVAL innehåller data för alla energier för k-rummet.
    \item OUTCAR innehåller alla utdata.   
    \item XDATCAR innehåller data om enhetscell, atompositioner för varje beräkningssteg och även atomstyp. 
    \item CONTCAR innehåller data som den återfunnen i POSCAR, men innehåller information om atompositioner uppdateras.  
    \item PCDAT Innehåller data för parkorrelationsfunktionen, PKF.  
\end{itemize}

\textbf{Indatafiler:} 
\begin{itemize}
    \setlength\itemsep{0em}
    \item INCAR innehåller information, i form av flaggor över hur beräkningar ska ske.
    \item POSCAR innehåller data om enhetcellen och atompositionering. 
    \item POTCAR innehåller data om atomtyper.
\end{itemize}

\newpage
Vid exempelvis beräkning av PKF för Si i temperaturen 300K, specificeras information om hur systemet ser ut i filer som POSCAR. Sedan kan information om hur beräkningarna ska genomföras specificeras i exempelvis INCAR eller POTCAR. Detta kan röra sig om hur många iterationer som ska ske och i vilka avstånd PKF ska beräknas. Då kan exempelvis flaggor som NPACO och APACO sättas i INCAR-filen. Där flaggan NPACO specificerar hur många iterationer som sker och APACO bestämmer det längsta avståndet som sista iteration ska ha.

Efter beräkningen genereras flera utdatafiler, däribland PCDAT, som innehåller värdena av PKF. Utdatafilen, PCDAT, kan då ha följande utseende:
\nyBild{PCDAT_utseende.pdf}{En demonstrativ bild över utseendet för PCDAT från VASP. Notera att värdena inte riktigt stämmer.}{PCDAT-utseende}{0.5}

Bilden \ref{fig:PCDAT-utseende} beskriver utseendet hos en del av PCDAT-filen för PKF för systemet Si i 300K, med 40 olika tidsteg. Viktigaste är den långa kolumnen av siffror som utgör definitionsmängden till funktionen. 

\subsubsection{HDF5-format} \label{sssec:rotgruppstr}
Vid hantering av stora mängder data, sådana genererade av beräkningsprogram som VASP, är HDF5-formatet mycket användbart. Det gör specificiering av olika dataförhållanden och beroenden enkla, samt tillgängliggör bearbetning av delar av data åt gången.   

En HDF5-fil är ett objekt som innehåller en rotgrupp, som äger alla andra grupper under den. Denna rotgrupp kan symboliseras av $"$\textit{/}$"$. Exempelvis $"$\textit{/foo/zoo}$"$ symboliserar \textit{zoo} som är en medlem till \textit{group} \textit{foo}, 
som vidare är en medlem till rotgruppen. 
Ett \textit{dataset} kan pekas av flera \textit{groups}. \cite{High Level Introduction to HDF5}

\nyBild{DemonstrativHDF5bild.png}{Schematisk bild över HDF5 struktur}{}{0.55}

Mer ingående består \textit{dataset}-objektet av metadata och rådata. Metadata beskriver rådatan, till den ingår \textit{dataspace}, \textit{datatype}, \textit{properties} och \textit{attributes}. Alla dessa är HDF5-objekt som beskriver olika saker. 

\textit{datatype} beskriver vad för datatyp varje individuell dataelement i ett dataset har. Exempelvis kan detta vara ett 32-bitars heltal, eller ett 32-bitars flyttal. I det mer komplexa fallet kan det också vara en sammansättning av flera, vanligt benämnda, datatyper. \textit{Datatype} beskriver då en följd av olika datatyper. Exempelvis en sammansättning som int16, char, int32, 2x3x2 array av 32-bit floats beskriver att varje dataelement i det gällande datasetet har en datatyp som består av 16 bitars heltal, en bokstav, 32-bitars heltal och slutligen en array av flyttal med dimensionen 2x3x2. \textit{dataspace} är en HDF5-objekt som beskriver hur datasetet sparar sin data, den kan exempelvis vara tom. Ett dataset kan även bestå av ett enda tal, eller vara en array. 
\newpage
\textit{Properties} är mindre konkret än de två tidigare nämnda egenskaperna och beskriver minneshanteringen av ett dataset. I dess defaultläge exempelvist är dataset sparade kontinuerligt. Slutligen återfinns HDF5-objektet \textit{attributes}, som kan valbart skapas. Typiskt sätt skapas \textit{attributes} som ett sätt för att ytterligare beskriva några egenskaper hos ett dataset. En \textit{attribute} innehåller ett namn och ett värde, och skapas i samband med att en dataset öppnas. \cite{HDF group2}       


\nyBild{Dataset_Metadata_HDF5.png}{Schematisk bild över \textit{dataset}.}{}{1}

ENVISIoN arbetar med HDF5-formatet. Python ger tillgång till hantering av HDF5-formatet via paketet \textit{h5py}. Detta tillgängliggör exempelvis läsandet av specifika element i massiva arrayer med användandet av syntaxer tillgängliga av paketet \textit{numpy}. \cite{How To Use HDF5 Files in Python}

Paketet \textit{h5py} ger upphov till HDF5-filer vilket kan ses som behållare för två sorters objekt, \textit{datasets} och \textit{groups}. \textit{datasets} är array-liknande ihopsättning av data, medan \textit{groups} fungerar som behållare för andra \textit{groups} eller \textit{datasets} \cite{How To Use HDF5 Files in Python}. Elementen i \textit{datasets} kan vara komplexa objekt. \textit{groups} kan återfinnas i andra \textit{groups}, detta ger därmed möjlighet till konstruktion av grupperingar av olika sammanhängande data. \textit{groups} och medlemmarna till \textit{groups} fungerar som mappar och filer i UNIX. Varje \textit{dataset} karaktäriseras exempelvis av en sökväg. \cite{High Level Introduction to HDF5}

\newpage
\subsection{ENVISIoNs HDF5-fil}
ENVISIoNs parsersystem använder sig av pythonmodulen \textit{h5py} för att generera en lämplig HDF5-filstruktur vid parsning. Den HDF5-strukturen som genereras återfinns i nedstående diagram. Notera att figuren visas som helbild i Appendix \ref{sec:appendixHDF5}.

\nyBild{UPDATE-hdf5-dataformat3modi.png}{En bild över HDF5-filstruktur som används i ENVISIoN.}{ENVISIoNsHDF5}{0.2}

I diagram \ref{fig:ENVISIoNsHDF5} nedan representeras olika grupper (\textit{groups}) av lådor med pilar (förutom lådorna vars brödtext är angiven i parantes), de sista lådorna i slutet av varje förgrening representerar olika \textit{dataset}. Diagrammet beskriver alltså hur information struktureras i en HDF5-fil som parsersystemet skapat. För att få tillgång till ett visst \textit{dataset} måste en sökväg anges. Denna sökväg är inget mer än en sträng bestående av olika grupper som beskriver hur ett \textit{dataset} nås från rotgruppen, se under rubrik \ref{sssec:rotgruppstr}. 

Varje \textit{dataset} kan bestå av ett antal olika fältnamn. Det fältnamn som alltid förekommer är \textit{value}, vilket beskriver den huvudsakliga datan som datasetet innehåller. Utöver det kan vissa andra fältnamn också förekomma, exempelvis "VariableName" vilket är olika attribut, \textit{attributes}, som beskriver andra egenskaper hos \textit{dataset} som kan vara intressant. 

Notera att diagram \ref{fig:ENVISIoNsHDF5} saknar viss information för DOS. DOS står för Density of States, översatt till tillståndstäthet. På grund av platsbrist har inte attributen skrivits ut för DOS. p-DOS, d-DOS(xy), Energy, grupper under DOS, med mera har attributen  

\begin{itemize}
 \item {\textbf{VariableName}} är fältets namn.
 \item{\textbf{VariableSymbol}} är en symbol som representerar variabeln. 
 \item{\textbf{QuantityName}} är ett för en människa läsligt namn på fältet. 
 \item{\textbf{QuantitySymbol}} är symbol som representerar storheten.
 \item{\textbf{Unit}} är storhetens fysikaliska enhet.
\end{itemize}

Notera också att \textit{float[x]} avser en lista med längd x, samt att alla grupper som är märkta med n är en metod att ange att det kan finns flera grupper på den nivån. Lådor vars rubrik är angivet inom parentes anger ett villkor för att den resterande sökvägen ska kunna skapas. Viktig anmärkning här är därför att dessa villkor inte ingår i HDF5-strukturen, de är inga grupper, och ingår därmed inte med sökvägen till de respektive dataseten. Under \textit{\/DOS} förekommer exempelvis en sådan låda med brödtexten \textit{(LORBIT=0)}, samt under förgreningen hos \textit{\/DOS\/Partial} förekommer en låda med angivelsen \textit{(ISPIN=0)}. Båda \textit{ISPIN} och \textit{LORBIT} är flaggor som kan sättas i INCAR-filen. I detta fall anger lådorna villkoren att \textit{(LORBIT=0)} och \textit{(ISPIN=0)} för att den fortsatta respektive grupperna under ska kunna skapas. Lådan under \textit{\/PairCorrelationFunc} anger dock ingen sådan flaggan. Det den anger är villkoret som har med huruvida \textit{\_write\_pcdat\_onecol} eller \textit{\_write\_pcdat\_multicol} används.

Parsning av PKF ges av olika möjligheter, parsern behandar en av följande fall: 

\begin{enumerate}
    \item System av flera atomtyper, det som beräknas är en genomsnittlig PKF över alla atomtyper.  
    \item System av flera atomtyper, det som beräknas är en genomsnittlig PKF för varje atomtyp. Ingår det K atomtyper i systemet ska parsern ge upphov till K stycken parkorrelationsfunktioner.
    \item System av 1 atomtyp. 
\end{enumerate}

För fall 2 och 3 används \textit{\_write\_pcdat\_multicol} medan fall 1 använder \textit{\_write\_pcdat\_onecol}, se under rubrik \ref{ssec:skrivning till HDF5}. Villkoren är därmed enbart ett sätt att ange vad för fall parsern behandlar. 
\newpage 

\subsection{Skrivning till HDF5-fil} \label{ssec:skrivning till HDF5}
Det som skapar strukturen i HDF5-filen är skrivningsmodulen \textit{h5writer} I ENVISIoN. \textit{h5writer.py} är ett skript som innehåller alla skrivningsfunktioner som ingår i parsersystemet. Funktionernas uppgift är att skapa \textit{datasets} (rådata) i rätt plats i HDF5-fil objektet. Nedan listas alla funktioner som ingår i modulen.

\textbf{\_write\_coordinates}
\newline 
Denna funktion skriver koordinater för atompositioner där varje atomslag tilldelas ett eget \textit{dataset}. Attribut sätts för respektive grundämnesbeteckning per \textit{dataset}.

Parametrar:
\begin{itemize}
    \setlength\itemsep{0em}
    \item h5file: Sökväg till HDF5-fil, anges som en sträng. 
    \item atom\_count: Lista med antalet atomer av de olika atomslagen.
    \item coordinates\_list: Lista med koordinater för samtliga atomer.     
    \item Elements: None eller lista med atomslag.      
\end{itemize}

Returnerar: 
\begin{itemize}
    \item None 
\end{itemize}

\textbf{\_write\_basis}
\newline 
Denna funktion skriver gittervektorerna i ett dataset med namn basis.

Parametrar:
\begin{itemize}
    \setlength\itemsep{0em}
    \item h5file: Sökväg till HDF5-fil, anges som en sträng.
    \item basis: Lista med basvektorerna.
\end{itemize}

Returnerar: 
\begin{itemize}
    \item None 
\end{itemize}

\textbf{\_write\_bandstruct}
\newline 
Denna funktion skriver ut data för bandstruktur i en grupp med namn Bandstructure. Inom denna grupp tilldelas specifika K-punkter, energier samt bandstrukturer egna dataset. Diverse attribut sätts även för bl.a. specifika energier.

Parametrar:
\begin{itemize}
    \setlength\itemsep{0em}
    \item h5file: Sökväg till HDF5-fil, anges som en sträng. 
    \item band\_data: Lista med bandstrukturdata.
    \item kval\_list: Lista med K-punkter för specifika bandstrukturdata.
\end{itemize}

Returnerar: 
\begin{itemize}
    \item None 
\end{itemize}

\textbf{\_write\_dos}
\newline 
Denna funktion skriver ut DOS-data i en grupp med namn DOS där total och partiell DOS tilldelas grupper med namn Total respektive Partial. Inom gruppen Total tilldelas energin samt specifika DOS egna dataset och inom gruppen Partial tilldelas varje partiell DOS egna grupper där energin samt specifika DOS tilldelas egna dataset.

Parametrar:
\begin{itemize}
    \setlength\itemsep{0em}
    \item h5file: Sökväg till HDF5-fil, anges som en sträng. 
    \item total: En lista med strängar av de olika uträkningarna som har utförts av VASP för total DOS.
    \item partial: En lista med strängar av de olika uträkningarna som har utförts av VASP för partiell DOS.
    \item total\_data: En lista med alla beräkningar för total DOS för varje specifik atom.
    \item partial\_list: En lista med alla beräkningar för partiell DOS för varje specifik atom.
    \item fermi\_energy: Fermi-energin för den aktuella uträkningen.
\end{itemize}

Returnerar: 
\begin{itemize}
    \item None 
\end{itemize}

\textbf{\_write\_volume}
\newline 
Denna funktion skriver ut elektrontäthetsdata och elektronlokaliseringsfunktionsdata (ELF) till grupper med namn CHG respektive ELF. Inom dessa grupper tilldelas varje iteration ett dataset.

Parametrar:
\begin{itemize}
    \setlength\itemsep{0em}
    \item h5file: Sökväg till HDF5-fil, anges som en sträng. 
    \item i: Skalär som anger numret på iterationen.
    \item partial:  En lista med strängar av de olika uträkningarna som har utförts av VASP för partiell DOS.
    \item array: Array med parsad data för respektive iteration.
    \item data\_dim: Lista som anger dimensionen av data för respektive iteration.
    \item hdfgroup: En textsträng med namnet på vad man vill kalla gruppen i HDF5-filen.
\end{itemize}

Returnerar: 
\begin{itemize}
    \item None 
\end{itemize}

\textbf{\_write\_incar}
\newline 
Denna funktion skriver ut parsad data från INCAR i ett dataset med namn Incar där varje datatyp tilldelas egna dataset.

Parametrar:
\begin{itemize}
    \setlength\itemsep{0em}
    \item h5file: Sökväg till HDF5-fil, anges som en sträng. 
    \item incar\_data: Datalexikon med all data från INCAR-filen.
\end{itemize}

Returnerar: 
\begin{itemize}
    \item None 
\end{itemize}

\textbf{\_write\_pcdat\_onecol}
\newline 
Denna funktion skapar ett HDF5-struktur för ett system med flera atomtyper, där en genomsnittlig PKF beräknas för alla atomtyper. Funktionen skapar en HDF5-struktur som innehåller data från huvudsakligen PCDAT. 

Parametrar:
\begin{itemize}
    \setlength\itemsep{0em}
    \item h5file: Sökväg till HDF5-fil, anges som en sträng. 
    \item pcdat\_data: Tillhör Python-datatypen \textit{dictionary} \cite{dict}. Detta argument innehåller alla värden av PKF som parsats.  
    \item APACO\_val: Värdet på APACO-flaggan i VASP-filen INCAR eller POTCAR. Defaultvärde är 16 Ångström. Flaggan anger det längsta avståndet sista iteration för beräkning av PKF har.     
    \item NPACO\_val: Värdet på NPACO-flaggan i VASP-filen INCAR eller POTCAR. Defaultvärde är 256. Flaggan anger hur många iterationer ska ske för beräkning av PKF.      
\end{itemize}

Returnerar: 
\begin{itemize}
    \item None 
\end{itemize}

\textbf{\_write\_pcdat\_multicol}
\newline 
Denna funktion skapar ett HDF5-struktur för ett system med flera atomtyper, där en genomsnittlig PKF beräknas för varje atomtyp som ingår i systemet. Funktionen anropas också i fallet då systemet enbart består av en atomtyp. Funktionen skapar en HDF5-struktur som innehåller data från huvudsakligen PCDAT. 

Parameterar: 
\begin{itemize}
    \setlength\itemsep{0em}
    \item h5file: Sökväg till HDF5-fil, anges som en sträng.
    \item pcdat\_data: Tillhör Python-datatypen \textit{dictionary} \cite{dict}. Detta argument innehåller alla värden av PKF som parsats.
    \item APACO\_val: Värdet på APACO-flaggan i VASP-filen INCAR eller POTCAR. Defaultvärde är 16 Ångström. Flaggan anger det längsta avståndet sista iteration för beräkning av PKF har.     
    \item NPACO\_val: Värdet på NPACO-flaggan i VASP-filen INCAR eller POTCAR. Defaultvärde är 256. Flaggan anger hur många iterationer ska ske för beräkning av PKF.      
\end{itemize}

Returnerar: 
\begin{itemize}
    \item None 
\end{itemize}

\subsection{Inläsning av VASP-filer} \label{ssec:inläsning av VASP}
Innan en funktion kan skriva till HDF5-objektet krävs det att rätt inläsning av innehåll från relevant VASP-fil har skett. Detta är vad de olika läsningsfunktionerna i parsersystemet gör. Typiskt återfinns en pythonmodul för varje egenskap hos ett system som ska parsas. Nedan listas alla sådana moduler.

\subsubsection{Incarparser}
Incarparsern består av en pythonfil med namnet incar som innehåller funktionerna, incar och par- se\_incar. Dessa funktioner läser in och sparar information från INCAR-filen samt anropar en separat pythonmodul som skriver en HDF5-fil. 

Funktionen incar kontrollerar att HDF5-filen redan innehåller INCAR-data och anropar funktionen parse\_incar om så inte är fallet. Existerar INCAR-filen i användarens VASP-katalog parsas data av funktionen parse\_incar som då sparar ett dataset för varje datatyp och namnger dataseten därefter. Funktionen incar anropar sedan pythonmodulen som skriver HDF5-filen där varje enskilt \textit{dataset} tilldelas en egen grupp. 

Funktionsanrop: envision.parser.vasp.incar(h5file, vasp\_dir)

Parameterar: 
\begin{itemize}
    \setlength\itemsep{0em}
    \item h5file: Sökväg till HDF5-fil, anges som en sträng.
    \item vasp\_dir: Sökväg till VASP-katalog, anges som en sträng.      
\end{itemize}

Returnerar: 
\begin{itemize}
    \setlength\itemsep{0em}
    \item Lista med namn på data (\textit{datasets}) som parsas.
    \item Bool: True om parsning skett felfritt, False annars.
\end{itemize}

\subsubsection{Volymparser}
Volymparsern består av en mängd funktioner i en pythonfil som används för parsning av CHG och ELFCAR. Den kan läsa in och spara data på HDF5-format från båda dessa filer genom att anropa en pythonmodul. Detta är för att CHG och ELFCAR har samma struktur och består av ett antal iterationer av volymdata från volymberäkningar. Således innehåller den sista iterationen data som är mest korrekt. Därför skapar volymparsern också en länk till den sista iterationen i HDF5-filen för att data av högst kvalitet lätt ska kunna plockas ut.

Funktionsanrop vid parsning av CHG-data: envision.parser.vasp.charge(h5file, vasp\_dir)

Funktionsanrop vid parsning av ELFCAR-data: envision.parser.vasp.elf(h5file, vasp\_dir)

Parameterar: 
\begin{itemize}
    \setlength\itemsep{0em}
    \item h5file: Sökväg till HDF5-fil, anges som en sträng.
    \item vasp\_dir: Sökväg till VASP-katalog.      
\end{itemize}

Returnerar: 
\begin{itemize}
    \item Bool: True om parsning skett felfritt, False annars.
\end{itemize}

\subsubsection{Tillståndstäthetsparser}
Tillståndstäthetsparsern består av en mängd funktioner i en pythonfil som används för parsning av DOSCAR. DOSCAR-filen består först av den totala tillståndstätheten och sedan partiell till- ståndstäthet för varje atom i kristallen. Beroende på vad som står i INCAR kan dock denna data se väldigt olika ut. Flaggorna ISPIN, RWIGS och LORBIT i INCAR-filen avgör vad som skrivs i DOSCAR-filen. ISPIN-flaggan informerar om spinn har tagits hänsyn till vid beräkningar, RWIGS-flaggan specificerar Wigner-Seitz-radien för varje atomtyp och LORBIT-flaggan (kombinerat med RWIGS) avgör om PROCAR- eller PROOUT-filer (som DOSCAR-filen refererar till) skrivs. Parsern läser därför från data givet av incarparsern i HDF5-filen för att se hur DOSCAR ska parsas. Parsern delar upp data i två grupper i HDF5-filen, total och partiell. I gruppen partiell finns det en grupp för varje atom. Ett dataset för varje undersökt fenomen skrivs sedan ut för varje atom under partiell, och för total tillståndstäthet under total.

Funktionsanrop: envision.parser.vasp.dos(h5file, vasp\_dir)

Parameterar: 
\begin{itemize}
    \setlength\itemsep{0em}
    \item h5file: Sökväg till HDF5-fil, anges som en sträng.
    \item vasp\_dir: Sökväg till VASP-katalog.      
\end{itemize}

Returnerar: 
\begin{itemize}
    \item Bool: True om parsning skett felfritt, False annars.
\end{itemize}

\subsubsection{Enhetscellsparser}
Enhetscellparsern läser in gittervektorer, som multipliceras med skalfaktorn och skrivs till /basis i HDF5-filen. Atompositioner läses från POSCAR och om dessa är angivna med kartesiska koordinater räknas de om till koordinater med gittervektorerna som bas. Koordinaterna skrivs till HDF5-filen uppdelade efter atomslag och attribut sätts med respektive grundämnesbeteckning. Om dessa inte ges med parametern elements letar parsern i första hand i POTCAR och i andra hand i POSCAR.

Funktionsanrop: envision.parser.vasp.unitcell(h5file, vasp\_dir, elements = None)

Parameterar: 
\begin{itemize}
    \setlength\itemsep{0em}
    \item h5file: Sökväg till HDF5-fil, anges som en sträng.
    \item vasp\_dir: Sökväg till VASP-katalog.
    \item elements = None: None eller lista med atomslag. 
\end{itemize}

Returnerar: 
\begin{itemize}
    \item Bool: True om parsning skett felfritt, False annars.
\end{itemize}

\subsubsection{Parkorrelationsfunktionsparser}
Parkorrelationsfunktionsparser använder sig av ett antal olika funktioner, vilka alla anropas med funktionen \textit{paircorrelation(h5file, vasp\_dir)}. Parsningen görs genom inläsning av korrekt data från PCDAT-filen, samt inläsning av flaggor som NPACO och APACO. Parsen letar efter dessa flaggor i INCAR eller POTCAR för att se om de är satta. I fallet de inte är det antas deras defaultvärden.      

Funktionsanrop: envision.parser.vasp.paircorrelation(h5file, vasp\_dir) 

Parameterar: 
\begin{itemize}
    \setlength\itemsep{0em}
    \item h5file: Sökväg till HDF5-fil, anges som en sträng.
    \item vasp\_dir: Sökväg till VASP-katalog.
\end{itemize}

Returnerar: 
\begin{itemize}
    \item Bool: True om parsning skett felfritt. Ett undantag kan kastas om PCDAT-fil inte hittas.
\end{itemize}


\subsubsection{parse\_all}
parse\_all är en funktion för parsning av allt som finns i katalogen som ges som inparameter. Funktionen kallar på alla systemets parsers och skriver ut meddelande om vad som parsas och om parsningen gjordes eller ej.

Funktionsanrop: envision.parse\_all(h5\_path, dir)

Parameterar: 
\begin{itemize}
    \setlength\itemsep{0em}
    \item h5\_path: Sökväg till HDF5-fil, anges som en sträng.
    \item vasp\_dir: Sökväg till katalog med utdata-filer från beräkningsprogram.
\end{itemize}

Returnerar: 
\begin{itemize}
    \item Bool: True om parsning skett felfritt, False annars.
\end{itemize}

\subsection{Testning}
För att varje års projekt ska kunna kontrollera att alla parsersystem fungerar är det viktigt med testfiler. Detta kan också ge inblick i hur parsern är tänkt att fungera. En generell testmapp i ENVISIONs filstruktur för parsersystemet finns. Mappen innehåller för tillfället enbart tester för parsersystemet för PKF (det är både skrivning och läsningsfunktioner som testas). Testfiler för PKF parsern skapades med hjälp av pythonmodulen \textit{unittest} \cite{Unittest}. Detta test testar bland annat undantagshanteringen och viktiga returvärden hos olika funktioner hos parsersystemet för PKF. Testet kontrollerar exempelvis att parsersystemet kan hantera PCDAT-filer av olika utseenden. 

Test för parsersystemet för PKF har implementerats med en testfil med namnet \textit{test\_paircorrelation.py} samt en mapp vid namn \textit{testdata}. I \textit{testdata} finns det olika mappar med VASP-filer för olika system, som därmed testar att parsern fungerar korrekt för olika filer. Det är tanken att framtida utvecklare använder sig av denna mapp för att lägga in tester för nyskapade funktioner för parsning av någon ny egenskap. 


 



